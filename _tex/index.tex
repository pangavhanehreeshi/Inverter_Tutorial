% Options for packages loaded elsewhere
\PassOptionsToPackage{unicode,linktoc=all}{hyperref}
\PassOptionsToPackage{hyphens}{url}
\PassOptionsToPackage{dvipsnames,svgnames,x11names}{xcolor}
%
\documentclass[
  letterpaper,
  DIV=11,
  numbers=noendperiod]{scrartcl}

\usepackage{amsmath,amssymb}
\usepackage{iftex}
\ifPDFTeX
  \usepackage[T1]{fontenc}
  \usepackage[utf8]{inputenc}
  \usepackage{textcomp} % provide euro and other symbols
\else % if luatex or xetex
  \usepackage{unicode-math}
  \defaultfontfeatures{Scale=MatchLowercase}
  \defaultfontfeatures[\rmfamily]{Ligatures=TeX,Scale=1}
\fi
\usepackage{lmodern}
\ifPDFTeX\else  
    % xetex/luatex font selection
    \setmainfont[]{Calibri}
\fi
% Use upquote if available, for straight quotes in verbatim environments
\IfFileExists{upquote.sty}{\usepackage{upquote}}{}
\IfFileExists{microtype.sty}{% use microtype if available
  \usepackage[]{microtype}
  \UseMicrotypeSet[protrusion]{basicmath} % disable protrusion for tt fonts
}{}
\makeatletter
\@ifundefined{KOMAClassName}{% if non-KOMA class
  \IfFileExists{parskip.sty}{%
    \usepackage{parskip}
  }{% else
    \setlength{\parindent}{0pt}
    \setlength{\parskip}{6pt plus 2pt minus 1pt}}
}{% if KOMA class
  \KOMAoptions{parskip=half}}
\makeatother
\usepackage{xcolor}
\usepackage[top=30mm,left=20mm,heightrounded]{geometry}
\setlength{\emergencystretch}{3em} % prevent overfull lines
\setcounter{secnumdepth}{5}
% Make \paragraph and \subparagraph free-standing
\makeatletter
\ifx\paragraph\undefined\else
  \let\oldparagraph\paragraph
  \renewcommand{\paragraph}{
    \@ifstar
      \xxxParagraphStar
      \xxxParagraphNoStar
  }
  \newcommand{\xxxParagraphStar}[1]{\oldparagraph*{#1}\mbox{}}
  \newcommand{\xxxParagraphNoStar}[1]{\oldparagraph{#1}\mbox{}}
\fi
\ifx\subparagraph\undefined\else
  \let\oldsubparagraph\subparagraph
  \renewcommand{\subparagraph}{
    \@ifstar
      \xxxSubParagraphStar
      \xxxSubParagraphNoStar
  }
  \newcommand{\xxxSubParagraphStar}[1]{\oldsubparagraph*{#1}\mbox{}}
  \newcommand{\xxxSubParagraphNoStar}[1]{\oldsubparagraph{#1}\mbox{}}
\fi
\makeatother

\usepackage{color}
\usepackage{fancyvrb}
\newcommand{\VerbBar}{|}
\newcommand{\VERB}{\Verb[commandchars=\\\{\}]}
\DefineVerbatimEnvironment{Highlighting}{Verbatim}{commandchars=\\\{\}}
% Add ',fontsize=\small' for more characters per line
\newenvironment{Shaded}{}{}
\newcommand{\AlertTok}[1]{\textcolor[rgb]{1.00,0.33,0.33}{\textbf{#1}}}
\newcommand{\AnnotationTok}[1]{\textcolor[rgb]{0.42,0.45,0.49}{#1}}
\newcommand{\AttributeTok}[1]{\textcolor[rgb]{0.84,0.23,0.29}{#1}}
\newcommand{\BaseNTok}[1]{\textcolor[rgb]{0.00,0.36,0.77}{#1}}
\newcommand{\BuiltInTok}[1]{\textcolor[rgb]{0.84,0.23,0.29}{#1}}
\newcommand{\CharTok}[1]{\textcolor[rgb]{0.01,0.18,0.38}{#1}}
\newcommand{\CommentTok}[1]{\textcolor[rgb]{0.42,0.45,0.49}{#1}}
\newcommand{\CommentVarTok}[1]{\textcolor[rgb]{0.42,0.45,0.49}{#1}}
\newcommand{\ConstantTok}[1]{\textcolor[rgb]{0.00,0.36,0.77}{#1}}
\newcommand{\ControlFlowTok}[1]{\textcolor[rgb]{0.84,0.23,0.29}{#1}}
\newcommand{\DataTypeTok}[1]{\textcolor[rgb]{0.84,0.23,0.29}{#1}}
\newcommand{\DecValTok}[1]{\textcolor[rgb]{0.00,0.36,0.77}{#1}}
\newcommand{\DocumentationTok}[1]{\textcolor[rgb]{0.42,0.45,0.49}{#1}}
\newcommand{\ErrorTok}[1]{\textcolor[rgb]{1.00,0.33,0.33}{\underline{#1}}}
\newcommand{\ExtensionTok}[1]{\textcolor[rgb]{0.84,0.23,0.29}{\textbf{#1}}}
\newcommand{\FloatTok}[1]{\textcolor[rgb]{0.00,0.36,0.77}{#1}}
\newcommand{\FunctionTok}[1]{\textcolor[rgb]{0.44,0.26,0.76}{#1}}
\newcommand{\ImportTok}[1]{\textcolor[rgb]{0.01,0.18,0.38}{#1}}
\newcommand{\InformationTok}[1]{\textcolor[rgb]{0.42,0.45,0.49}{#1}}
\newcommand{\KeywordTok}[1]{\textcolor[rgb]{0.84,0.23,0.29}{#1}}
\newcommand{\NormalTok}[1]{\textcolor[rgb]{0.14,0.16,0.18}{#1}}
\newcommand{\OperatorTok}[1]{\textcolor[rgb]{0.14,0.16,0.18}{#1}}
\newcommand{\OtherTok}[1]{\textcolor[rgb]{0.44,0.26,0.76}{#1}}
\newcommand{\PreprocessorTok}[1]{\textcolor[rgb]{0.84,0.23,0.29}{#1}}
\newcommand{\RegionMarkerTok}[1]{\textcolor[rgb]{0.42,0.45,0.49}{#1}}
\newcommand{\SpecialCharTok}[1]{\textcolor[rgb]{0.00,0.36,0.77}{#1}}
\newcommand{\SpecialStringTok}[1]{\textcolor[rgb]{0.01,0.18,0.38}{#1}}
\newcommand{\StringTok}[1]{\textcolor[rgb]{0.01,0.18,0.38}{#1}}
\newcommand{\VariableTok}[1]{\textcolor[rgb]{0.89,0.38,0.04}{#1}}
\newcommand{\VerbatimStringTok}[1]{\textcolor[rgb]{0.01,0.18,0.38}{#1}}
\newcommand{\WarningTok}[1]{\textcolor[rgb]{1.00,0.33,0.33}{#1}}

\providecommand{\tightlist}{%
  \setlength{\itemsep}{0pt}\setlength{\parskip}{0pt}}\usepackage{longtable,booktabs,array}
\usepackage{calc} % for calculating minipage widths
% Correct order of tables after \paragraph or \subparagraph
\usepackage{etoolbox}
\makeatletter
\patchcmd\longtable{\par}{\if@noskipsec\mbox{}\fi\par}{}{}
\makeatother
% Allow footnotes in longtable head/foot
\IfFileExists{footnotehyper.sty}{\usepackage{footnotehyper}}{\usepackage{footnote}}
\makesavenoteenv{longtable}
\usepackage{graphicx}
\makeatletter
\newsavebox\pandoc@box
\newcommand*\pandocbounded[1]{% scales image to fit in text height/width
  \sbox\pandoc@box{#1}%
  \Gscale@div\@tempa{\textheight}{\dimexpr\ht\pandoc@box+\dp\pandoc@box\relax}%
  \Gscale@div\@tempb{\linewidth}{\wd\pandoc@box}%
  \ifdim\@tempb\p@<\@tempa\p@\let\@tempa\@tempb\fi% select the smaller of both
  \ifdim\@tempa\p@<\p@\scalebox{\@tempa}{\usebox\pandoc@box}%
  \else\usebox{\pandoc@box}%
  \fi%
}
% Set default figure placement to htbp
\def\fps@figure{htbp}
\makeatother

\KOMAoption{captions}{tableheading}
\makeatletter
\@ifpackageloaded{tcolorbox}{}{\usepackage[skins,breakable]{tcolorbox}}
\@ifpackageloaded{fontawesome5}{}{\usepackage{fontawesome5}}
\definecolor{quarto-callout-color}{HTML}{909090}
\definecolor{quarto-callout-note-color}{HTML}{0758E5}
\definecolor{quarto-callout-important-color}{HTML}{CC1914}
\definecolor{quarto-callout-warning-color}{HTML}{EB9113}
\definecolor{quarto-callout-tip-color}{HTML}{00A047}
\definecolor{quarto-callout-caution-color}{HTML}{FC5300}
\definecolor{quarto-callout-color-frame}{HTML}{acacac}
\definecolor{quarto-callout-note-color-frame}{HTML}{4582ec}
\definecolor{quarto-callout-important-color-frame}{HTML}{d9534f}
\definecolor{quarto-callout-warning-color-frame}{HTML}{f0ad4e}
\definecolor{quarto-callout-tip-color-frame}{HTML}{02b875}
\definecolor{quarto-callout-caution-color-frame}{HTML}{fd7e14}
\makeatother
\makeatletter
\@ifpackageloaded{caption}{}{\usepackage{caption}}
\AtBeginDocument{%
\ifdefined\contentsname
  \renewcommand*\contentsname{Table of contents}
\else
  \newcommand\contentsname{Table of contents}
\fi
\ifdefined\listfigurename
  \renewcommand*\listfigurename{List of Figures}
\else
  \newcommand\listfigurename{List of Figures}
\fi
\ifdefined\listtablename
  \renewcommand*\listtablename{List of Tables}
\else
  \newcommand\listtablename{List of Tables}
\fi
\ifdefined\figurename
  \renewcommand*\figurename{Figure}
\else
  \newcommand\figurename{Figure}
\fi
\ifdefined\tablename
  \renewcommand*\tablename{Table}
\else
  \newcommand\tablename{Table}
\fi
}
\@ifpackageloaded{float}{}{\usepackage{float}}
\floatstyle{ruled}
\@ifundefined{c@chapter}{\newfloat{codelisting}{h}{lop}}{\newfloat{codelisting}{h}{lop}[chapter]}
\floatname{codelisting}{Listing}
\newcommand*\listoflistings{\listof{codelisting}{List of Listings}}
\makeatother
\makeatletter
\makeatother
\makeatletter
\@ifpackageloaded{caption}{}{\usepackage{caption}}
\@ifpackageloaded{subcaption}{}{\usepackage{subcaption}}
\makeatother

\usepackage{bookmark}

\IfFileExists{xurl.sty}{\usepackage{xurl}}{} % add URL line breaks if available
\urlstyle{same} % disable monospaced font for URLs
\hypersetup{
  pdftitle={Analog Inverter Tutorial},
  pdfauthor={Hrishikesh Pangavhane},
  colorlinks=true,
  linkcolor={blue},
  filecolor={Maroon},
  citecolor={Blue},
  urlcolor={Blue},
  pdfcreator={LaTeX via pandoc}}


\title{Analog Inverter Tutorial}
\author{Hrishikesh Pangavhane}
\date{2026-02-16}

\begin{document}
\maketitle
\begin{abstract}
In \ldots{} general \ldots{}
\end{abstract}

\renewcommand*\contentsname{Table of contents}
{
\hypersetup{linkcolor=}
\setcounter{tocdepth}{3}
\tableofcontents
}

\section{Introduction}\label{introduction}

This tutorial provides a step-by-step introduction to designing and
simulating a \textbf{simple analog inverter} using the
\textbf{open-source tools} hosted on the
\href{https://github.com/iic-jku/IIC-OSIC-TOOLS}{IIC-OSIC} platform,
specifically with the
\href{https://github.com/IHP-GmbH/IHP-Open-PDK}{IHP SG13G2 PDK}. The
analog inverter, though simple, is a crucial element in analog and
mixed-signal circuit design, offering valuable insights into device
behavior, biasing, and small-signal performance. The goal of this
tutorial is to prrovide a concise overview of the steps needed to create
and verify simple CMOS circuits and systems.

\section{Full design flow using IIC-OSIC
tools}\label{full-design-flow-using-iic-osic-tools}

\subsection{Schematic entry using
Xschem}\label{schematic-entry-using-xschem}

Schematic entry is performed using
\href{https://xschem.sourceforge.io/stefan/xschem_man/xschem_man.html}{Xschem},
a powerful open-source schematic editor well-suited for analog and
mixed-signal circuit design. In this step, the analog inverter circuit
is constructed by placing and connecting NMOS and PMOS transistors from
the \textbf{IHP SG13G2 PDK} library. Xschem provides an intuitive
interface for defining circuit topology, assigning device parameters
like W/L ratios, and labeling nodes for simulation. Proper hierarchy and
net labeling ensure compatibility with simulation and layout tools used
later in the design flow.

\begin{enumerate}
\def\labelenumi{\arabic{enumi}.}
\item
  Start terminal and run IIC-OSIC docker image
\item
  To select the PDK, use command \texttt{iic-pdk\ ihp-sg13g} refer
  Figure~\ref{fig-select-pdk}
\end{enumerate}

\begin{figure}

\centering{

\pandocbounded{\includegraphics[keepaspectratio]{./figures/_fig_choose_pdk.png}}

}

\caption{\label{fig-select-pdk}Xschem Entry.}

\end{figure}%

\begin{enumerate}
\def\labelenumi{\arabic{enumi}.}
\setcounter{enumi}{2}
\item
  Start xschem and create a new schematic. Name it as
  (Analog\_Inverter.sch)
\item
  Using \texttt{Ctrl+i} insert symbol from the sg13g2 library for
  lv\_nmos and lv\_pmos where lv stands for low-voltage
\item
  customize the transistor's dimensions (select the symbol, right click
  on it and edit its properties) refer Figure~\ref{fig-edit-properties}
\end{enumerate}

\begin{figure}

\centering{

\pandocbounded{\includegraphics[keepaspectratio]{./figures/_fig_edit_properties.png}}

}

\caption{\label{fig-edit-properties}Xschem Entry.}

\end{figure}%

\begin{enumerate}
\def\labelenumi{\arabic{enumi}.}
\setcounter{enumi}{5}
\item
  Place an input pin \texttt{ipin.sym} and an output pin
  \texttt{opin.sym} the symbols are available in the library
  \texttt{\textasciitilde{}/foss/tools/xschem/share/xschem/xschem\_library/devices}
  label the name of the input and output pins (select the pin, right
  click on it and edit its properties)
\item
  To wire to components, \texttt{Ctrl+w}
\item
  Once finished, save the schematic \texttt{Ctrl+Shift+S} See
  Figure~\ref{fig-xschem-symbol}
\end{enumerate}

\begin{figure}

\centering{

\pandocbounded{\includegraphics[keepaspectratio]{./figures/Analog_Inverter_xschem.png}}

}

\caption{\label{fig-xschem-entry}Xschem Entry.}

\end{figure}%

\begin{enumerate}
\def\labelenumi{\arabic{enumi}.}
\setcounter{enumi}{8}
\tightlist
\item
  For creating a symbol,
  \texttt{Symbol\ -\/-\textgreater{}\ Make\ symbol\ from\ schematic\ \textbar{}\ Ctrl+L}.
  We can also edit the symbol by opening it in a new tab, see
  Figure~\ref{fig-xschem-symbol}
\end{enumerate}

\begin{figure}

\centering{

\pandocbounded{\includegraphics[keepaspectratio]{index_files/mediabag/figures/Analog_Inverter_symbol.pdf}}

}

\caption{\label{fig-xschem-symbol}Xschem Symbol.}

\end{figure}%

\begin{enumerate}
\def\labelenumi{\arabic{enumi}.}
\setcounter{enumi}{9}
\tightlist
\item
  Now that we have created a symbol, we need to perform DC and Transient
  analysis using NgSpice.
\end{enumerate}

\subsection{Simulation using NgSpice}\label{simulation-using-ngspice}

To perform various analysis, we need to create simulation testbenches.

\begin{enumerate}
\def\labelenumi{\arabic{enumi}.}
\tightlist
\item
  Open new tab in xschem \texttt{Ctrl+o}, insert the symbol for inverter
  (Analog\_Inverter.sym) \texttt{Ctrl+i}. Design a testbench in xschem
  see Figure~\ref{fig-tb-dc}.
\end{enumerate}

\begin{figure}

\centering{

\pandocbounded{\includegraphics[keepaspectratio]{index_files/mediabag/figures/_fig_testbench.pdf}}

}

\caption{\label{fig-tb-dc}Xschem Testbench.}

\end{figure}%

\begin{enumerate}
\def\labelenumi{\arabic{enumi}.}
\setcounter{enumi}{1}
\tightlist
\item
  The spice directives \textbf{MODELS1} and \textbf{NGSPICE1} are
  code.sym instances with the attributes shown in
  Figure~\ref{fig-NgSpice} and Figure~\ref{fig-TT-Models}
\end{enumerate}

\begin{figure}

\centering{

\pandocbounded{\includegraphics[keepaspectratio]{./figures/_fig_NgSpice.png}}

}

\caption{\label{fig-NgSpice}Xschem Testbench.}

\end{figure}%

\begin{figure}

\centering{

\pandocbounded{\includegraphics[keepaspectratio]{./figures/_fig_TT_Models.png}}

}

\caption{\label{fig-TT-Models}Xschem TT Corner.}

\end{figure}%

\begin{enumerate}
\def\labelenumi{\arabic{enumi}.}
\setcounter{enumi}{2}
\tightlist
\item
  Click the \textbf{Netlist} button to create a netlist which will give
  us a \texttt{.spice} file and then click on \textbf{Simulate} button
  to run the simulation.
\end{enumerate}

\begin{tcolorbox}[enhanced jigsaw, coltitle=black, titlerule=0mm, arc=.35mm, opacitybacktitle=0.6, colframe=quarto-callout-note-color-frame, bottomrule=.15mm, colbacktitle=quarto-callout-note-color!10!white, breakable, opacityback=0, toprule=.15mm, leftrule=.75mm, bottomtitle=1mm, toptitle=1mm, colback=white, title=\textcolor{quarto-callout-note-color}{\faInfo}\hspace{0.5em}{Important}, rightrule=.15mm, left=2mm]

We have used NgSpice interactive mode for the spice simulation see
Figure~\ref{fig-spice-mode}. We can also find other modes
\texttt{Simulation\ -\/-\textgreater{}\ Configure\ simulator\ and\ tools}

\end{tcolorbox}

\begin{figure}

\centering{

\pandocbounded{\includegraphics[keepaspectratio]{./figures/_fig_NgSpice_Mode.png}}

}

\caption{\label{fig-spice-mode}Simulator Mode}

\end{figure}%

\subsection{Output Results}\label{output-results}

Outputs of dc and transient analysis can be seen in

\begin{figure}

\centering{

\pandocbounded{\includegraphics[keepaspectratio]{./figures/_fig_tb_dc.png}}

}

\caption{\label{fig-tb-dc1}DC Analysis}

\end{figure}%

\begin{figure}

\centering{

\pandocbounded{\includegraphics[keepaspectratio]{./figures/_fig_tb_tran.png}}

}

\caption{\label{fig-tb-tran}Transient Analysis}

\end{figure}%

\subsection{Post processing and simulation using
python}\label{post-processing-and-simulation-using-python}

\begin{figure}[H]

{\centering \pandocbounded{\includegraphics[keepaspectratio]{index_files/mediabag/figures/_fig_inverter_transient.pdf}}

}

\caption{Transient Curver}

\end{figure}%%
\begin{figure}[H]

{\centering \pandocbounded{\includegraphics[keepaspectratio]{index_files/mediabag/figures/_fig_inverter_dc_analysis.pdf}}

}

\caption{DC Charachteristics}

\end{figure}%%
\begin{figure}[H]

{\centering \pandocbounded{\includegraphics[keepaspectratio]{index_files/mediabag/figures/_fig_inverter_vtc.pdf}}

}

\caption{VTC}

\end{figure}%

\subsection{Introduction to Layout}\label{introduction-to-layout}

Layout is the physical representation of an integrated circuit, where
circuit components such as transistors, interconnects, and vias are
defined geometrically on various layers corresponding to materials used
in fabrication. The layout must adhere strictly to design rules provided
by the semiconductor foundry. We have two tools \textbf{Magic} and
\textbf{KLayout} which enables layout design and verification. Magic
being a long-standing tool favored for its tight integration with
open-source flows, while KLayout offers powerful editing, scripting, and
visualization capabilities, especially suited for hierarchical designs
and PDKs such as SG13G2.

\subsubsection{Designing Layout using
Klayout}\label{designing-layout-using-klayout}

Layout design is performed using
\href{https://github.com/KLayout/klayout}{KLayout}, an open-source
layout viewer and editor for GDSII and OASIS files. Inverter layout can
be built either \textbf{manually from scratch or using parametric cells
(P\_Cells)} from the IHP SG13G2 PDK, p\_cells simplify device
instantiation by allowing parameterized generation of transistors and
other structures. KLayout supports precise layer control, design rule
checking, and layout-versus-schematic (LVS) compatibility.

\begin{enumerate}
\def\labelenumi{\arabic{enumi}.}
\item
  Start terminal and run IIC-OSIC docker image.
\item
  To select the PDK, use command \texttt{iic-pdk\ ihp-sg13g} refer
  Figure~\ref{fig-select-pdk}
\end{enumerate}

\begin{figure}

\centering{

\pandocbounded{\includegraphics[keepaspectratio]{./figures/_fig_choose_pdk.png}}

}

\caption{\label{fig-select-pdk}Technology Selection}

\end{figure}%

\begin{tcolorbox}[enhanced jigsaw, coltitle=black, titlerule=0mm, arc=.35mm, opacitybacktitle=0.6, colframe=quarto-callout-note-color-frame, bottomrule=.15mm, colbacktitle=quarto-callout-note-color!10!white, breakable, opacityback=0, toprule=.15mm, leftrule=.75mm, bottomtitle=1mm, toptitle=1mm, colback=white, title=\textcolor{quarto-callout-note-color}{\faInfo}\hspace{0.5em}{Important}, rightrule=.15mm, left=2mm]

This step is crucial as it sets the technology IHP-SG13G2 in the tools.

\end{tcolorbox}

\begin{enumerate}
\def\labelenumi{\arabic{enumi}.}
\setcounter{enumi}{2}
\tightlist
\item
  Start klayout amd refer Figure~\ref{fig-klayout-start}. A new window
  of Klayout will open after the command.
\end{enumerate}

\begin{figure}

\centering{

\pandocbounded{\includegraphics[keepaspectratio]{./figures/_fig_klayout_start.png}}

}

\caption{\label{fig-klayout-start}Klayout Launch.}

\end{figure}%

\begin{enumerate}
\def\labelenumi{\arabic{enumi}.}
\setcounter{enumi}{3}
\tightlist
\item
  To create new layout, Go to
  \texttt{Files\ \textgreater{}\textgreater{}\ New\ Layout}. New Layout
  Properties dialog box will appear, change
  \texttt{Top\ Cell\ from\ TOP\ to\ inv(inverter)} and leave the rest
  the same refer Figure~\ref{fig-new-layout}
\end{enumerate}

\begin{figure}

\centering{

\pandocbounded{\includegraphics[keepaspectratio]{./figures/_fig_setting.png}}

}

\caption{\label{fig-new-layout}New Layout}

\end{figure}%

\begin{enumerate}
\def\labelenumi{\arabic{enumi}.}
\setcounter{enumi}{4}
\tightlist
\item
  Now go to \texttt{Basic\ -\ Basic\ Layout\ Objects} on the left panel.
  Under \texttt{SG13\_dev\ -\ IHP\ SG13G2\ Pcells}, select an NMOS
  device and place it in the layout. \texttt{Press\ ESC} to exit
  placement mode. Repeat the process for the PMOS, placing it directly
  above the NMOS to follow standard inverter structure.
\end{enumerate}

\begin{figure}

\centering{

\pandocbounded{\includegraphics[keepaspectratio]{./figures/_fig_p_cell.png}}

}

\caption{\label{fig-p-cell}Pcells}

\end{figure}%

\begin{enumerate}
\def\labelenumi{\arabic{enumi}.}
\setcounter{enumi}{5}
\tightlist
\item
  On the right side, the layer list shows all available layers; bold
  entries indicate layers currently used in the layout. To view all
  layers from the full cell hierarchy,
  \texttt{click\ Display\ →\ Full\ Hierarchy\ \textbar{}\ Press\ Shift\ +\ F}.
\end{enumerate}

\begin{tcolorbox}[enhanced jigsaw, coltitle=black, titlerule=0mm, arc=.35mm, opacitybacktitle=0.6, colframe=quarto-callout-note-color-frame, bottomrule=.15mm, colbacktitle=quarto-callout-note-color!10!white, breakable, opacityback=0, toprule=.15mm, leftrule=.75mm, bottomtitle=1mm, toptitle=1mm, colback=white, title=\textcolor{quarto-callout-note-color}{\faInfo}\hspace{0.5em}{Tips}, rightrule=.15mm, left=2mm]

To rotate a PCell, go to the
\texttt{Edit\ tab\ \textgreater{}\textgreater{}\ Move}, select the
PCell, then right-click. Use the mouse wheel to zoom in and out, and
press the middle mouse button to pan the layout view.

\end{tcolorbox}

\begin{enumerate}
\def\labelenumi{\arabic{enumi}.}
\setcounter{enumi}{6}
\tightlist
\item
  \texttt{Double-click\ on\ the\ PCell} to open its parameter editor.
  Set parameters like the width and length values to match those used in
  your \textbf{Xschem schematic}, then \texttt{click\ OK} to apply the
  changes refer Figure~\ref{fig-pcell-para}
\end{enumerate}

\begin{figure}

\centering{

\pandocbounded{\includegraphics[keepaspectratio]{./figures/_fig_pcell_parameters.png}}

}

\caption{\label{fig-pcell-para}Parameters Window}

\end{figure}%

\begin{enumerate}
\def\labelenumi{\arabic{enumi}.}
\setcounter{enumi}{7}
\item
  Use the keyboard arrow keys to align the NMOS directly below the PMOS.
  Connect the \textbf{input} by joining the gates of both transistors
  using the \texttt{polysilicon\ layer\ -\ GatPoly.drawing}. Connect the
  \textbf{output} by linking the source and drain of PMOS and NMOS with
  the \texttt{Metal1\ layer\ -\ Metal1.drawing}. Similarly, use
  \texttt{Metal1.drawing} to connect the PMOS source to VCC and the NMOS
  source to VSS. Label all metal connections using the
  \texttt{Metal1\ pin\ layer\ -\ Metal1.pin}. Refer to the
  \href{https://ihp-open-pdk-docs.readthedocs.io/en/latest/layout_rules/02_layer_table.html\#layer-table}{IHP
  Open Source PDK DesignLIB} documentation to verify the correct layer
  numbers and purposes.
\item
  Finally, save your layout by going to \texttt{File\ →\ Save\ As}, and
  choose the \texttt{GDSII} format to export your design. Your completed
  CMOS inverter layout should resemble the image shown below.
\end{enumerate}

\begin{figure}

\centering{

\pandocbounded{\includegraphics[keepaspectratio]{./figures/_fig_inverter.png}}

}

\caption{\label{fig-inv}Final Layout}

\end{figure}%

\href{https://github.com/IHP-GmbH/IHP-Open-PDK/blob/main/ihp-sg13g2/libs.doc/doc/SG13G2_os_layout_rules.pdf}{Here}
you can find all the layout rules in depth and with illustrations.

\section{Klayout-PEX (KPEX)}\label{klayout-pex-kpex}

While in the schematic, a connection between device terminals is seen as
an equipotential, the stacked geometries in a specific layout introduce
parasitic effects, which can be thought of additional resistors,
capacitors (and inductors), not modeled by and missing in the original
schematic.

To be able to simulate these effects, a parasitic extraction tool (PEX)
is used, to extract a netlist from the layout, which represents the
original schematic (created from the layout active and passive elements)
augmented with the additional parasitic devices.

\subsection{Why Do We Need PEX?}\label{why-do-we-need-pex}

\begin{longtable}[]{@{}
  >{\raggedright\arraybackslash}p{(\linewidth - 2\tabcolsep) * \real{0.3000}}
  >{\raggedright\arraybackslash}p{(\linewidth - 2\tabcolsep) * \real{0.7000}}@{}}
\toprule\noalign{}
\begin{minipage}[b]{\linewidth}\raggedright
Reason
\end{minipage} & \begin{minipage}[b]{\linewidth}\raggedright
Description
\end{minipage} \\
\midrule\noalign{}
\endhead
\bottomrule\noalign{}
\endlastfoot
\textbf{Performance Degradation} & Parasitic capacitance reduces
bandwidth and phase margin. \\
\textbf{Offset and Mismatch Effects} & Parasitic coupling causes
imbalance in differential paths. \\
\textbf{Stability Impact} & Additional phase lag from parasitics can
destabilize feedback loops. \\
\textbf{Post-Layout Verification} & Confirms that layout did not violate
design intent (gain, UGB, PSRR, etc.). \\
\textbf{Tapeout Confidence} & Ensures high correlation between
simulation and silicon behavior. \\
\end{longtable}

IIC-OSIC-Tools offers
\href{https://martinjankoehler.github.io/klayout-pex-website/doc/doc.html}{KPEX}
for this purpose which is a tool, well integrated with Klayout by using
its API.

\subsection{PEX using Klayout-PEX
tool}\label{pex-using-klayout-pex-tool}

IIC-OSIC-Tools comes with pre-installed tool called as \textbf{K-pex}.
The current status of
\href{https://martinjankoehler.github.io/klayout-pex-website/doc/doc.html\#sec-into-klayout-pex-prototype-status}{KLayout-PEX}
says as following:

Available KLayout PEX Engines:

\begin{longtable}[]{@{}
  >{\raggedright\arraybackslash}p{(\linewidth - 6\tabcolsep) * \real{0.1418}}
  >{\raggedright\arraybackslash}p{(\linewidth - 6\tabcolsep) * \real{0.0993}}
  >{\raggedright\arraybackslash}p{(\linewidth - 6\tabcolsep) * \real{0.1773}}
  >{\raggedright\arraybackslash}p{(\linewidth - 6\tabcolsep) * \real{0.5816}}@{}}
\toprule\noalign{}
\begin{minipage}[b]{\linewidth}\raggedright
\textbf{Engine}
\end{minipage} & \begin{minipage}[b]{\linewidth}\raggedright
\textbf{PEX Type}
\end{minipage} & \begin{minipage}[b]{\linewidth}\raggedright
\textbf{Status}
\end{minipage} & \begin{minipage}[b]{\linewidth}\raggedright
\textbf{Description}
\end{minipage} \\
\midrule\noalign{}
\endhead
\bottomrule\noalign{}
\endlastfoot
\texttt{KPEX/MAGIC} & CC, RC & Usable & Wrapper engine, using installed
\texttt{magic} tool \\
\texttt{KPEX/FasterCap} & CC & Usable, \emph{pending QA} & Field solver
engine using \texttt{FasterCap} \\
\texttt{KPEX/FastHenry2} & R, L & Planned & Field solver engine using
\texttt{FastHenry2} \\
\texttt{KPEX/2.5D} & CC & Under construction & Prototype engine
implementing MAGIC concepts/formulas with \texttt{KLayout} means \\
\texttt{KPEX/2.5D} & R, RC & Planned & Prototype engine implementing
MAGIC concepts/formulas with \texttt{KLayout} means \\
\end{longtable}

\subsection{\texorpdfstring{Running the \texttt{KPEX/MAGIC}
Engine}{Running the KPEX/MAGIC Engine}}\label{running-the-kpexmagic-engine}

The magic section of \texttt{kpex\ -\/-help} describes the arguments and
their defaults. Important arguments:

\begin{itemize}
\tightlist
\item
  \texttt{-\/-magicrc}: specify location of the \texttt{magicrc} file\\
\item
  \texttt{-\/-gds}: path to the GDS input layout\\
\item
  \texttt{-\/-magic}: enable magic engine
\item
  \texttt{-\/-out\_dir} : set the output directory
\end{itemize}

\subsubsection{Example Command}\label{example-command}

\begin{Shaded}
\begin{Highlighting}[]
\ExtensionTok{kpex} \AttributeTok{{-}{-}pdk}\NormalTok{ ihp\_sg13g2 }\AttributeTok{{-}{-}magic} \AttributeTok{{-}{-}gds}\NormalTok{ GDS\_PATH }\AttributeTok{{-}{-}out\_dir}\NormalTok{ OUTPUT\_DIR\_PATH}
\end{Highlighting}
\end{Shaded}

more to kpex can be found under the command \texttt{kpex\ -\/-help}

\begin{Shaded}
\begin{Highlighting}[]
\ExtensionTok{/foss/designs} \OperatorTok{\textgreater{}}\NormalTok{ kpex }\AttributeTok{{-}{-}help}
\ExtensionTok{Usage:}\NormalTok{ kpex }\PreprocessorTok{[{-}{-}}\SpecialStringTok{help}\PreprocessorTok{]} \PreprocessorTok{[{-}{-}}\SpecialStringTok{version}\PreprocessorTok{]}\NormalTok{ [{-}{-}log\_level LOG\_LEVEL]}
            \ExtensionTok{[{-}{-}threads}\NormalTok{ NUM\_THREADS] }\AttributeTok{{-}{-}pdk} \DataTypeTok{\{ihp\_sg13g2}\OperatorTok{,}\DataTypeTok{sky130A\}}
            \ExtensionTok{[{-}{-}out\_dir}\NormalTok{ OUTPUT\_DIR\_BASE\_PATH] [{-}{-}gds GDS\_PATH]}
            \ExtensionTok{[{-}{-}schematic}\NormalTok{ SCHEMATIC\_PATH] [{-}{-}lvsdb LVSDB\_PATH]}
            \ExtensionTok{[{-}{-}cell}\NormalTok{ CELL\_NAME] [{-}{-}cache{-}lvs CACHE\_LVS]}
            \ExtensionTok{[{-}{-}cache{-}dir}\NormalTok{ CACHE\_DIR\_PATH] [{-}{-}lvs{-}verbose KLAYOUT\_LVS\_VERBOSE]}
            \ExtensionTok{[{-}{-}blackbox}\NormalTok{ BLACKBOX\_DEVICES] }\PreprocessorTok{[{-}{-}}\SpecialStringTok{fastercap}\PreprocessorTok{]} \PreprocessorTok{[{-}{-}}\SpecialStringTok{fastcap}\PreprocessorTok{]} \PreprocessorTok{[{-}{-}}\SpecialStringTok{magic}\PreprocessorTok{]}
            \ExtensionTok{[{-}{-}2.5D]}\NormalTok{ [{-}{-}k\_void K\_VOID] [{-}{-}delaunay\_amax DELAUNAY\_AMAX]}
            \ExtensionTok{[{-}{-}delaunay\_b}\NormalTok{ DELAUNAY\_B] [{-}{-}geo\_check GEOMETRY\_CHECK]}
            \ExtensionTok{[{-}{-}diel}\NormalTok{ DIELECTRIC\_FILTER] [{-}{-}tolerance FASTERCAP\_TOLERANCE]}
            \ExtensionTok{[{-}{-}d\_coeff}\NormalTok{ FASTERCAP\_D\_COEFF]}
            \ExtensionTok{[{-}{-}mesh}\NormalTok{ FASTERCAP\_MESH\_REFINEMENT\_VALUE]}
            \ExtensionTok{[{-}{-}ooc}\NormalTok{ FASTERCAP\_OOC\_CONDITION]}
            \ExtensionTok{[{-}{-}auto\_precond}\NormalTok{ FASTERCAP\_AUTO\_PRECONDITIONER] }\PreprocessorTok{[{-}{-}}\SpecialStringTok{galerkin}\PreprocessorTok{]}
            \ExtensionTok{[{-}{-}jacobi]}\NormalTok{ [{-}{-}magicrc MAGICRC\_PATH] [{-}{-}magic\_mode }\DataTypeTok{\{CC}\OperatorTok{,}\DataTypeTok{RC}\OperatorTok{,}\DataTypeTok{R\}}\NormalTok{]}
            \ExtensionTok{[{-}{-}magic\_cthresh}\NormalTok{ MAGIC\_CTHRESH] [{-}{-}magic\_rthresh MAGIC\_RTHRESH]}
            \ExtensionTok{[{-}{-}magic\_tolerance}\NormalTok{ MAGIC\_TOLERANCE] [{-}{-}magic\_halo MAGIC\_HALO]}
            \ExtensionTok{[{-}{-}magic\_short} \DataTypeTok{\{none}\OperatorTok{,}\DataTypeTok{resistor}\OperatorTok{,}\DataTypeTok{voltage\}}\NormalTok{]}
            \ExtensionTok{[{-}{-}magic\_merge} \DataTypeTok{\{none}\OperatorTok{,}\DataTypeTok{conservative}\OperatorTok{,}\DataTypeTok{aggressive\}}\NormalTok{] [{-}{-}mode }\DataTypeTok{\{CC}\OperatorTok{,}\DataTypeTok{RC}\OperatorTok{,}\DataTypeTok{R\}}\NormalTok{]}
            \ExtensionTok{[{-}{-}halo}\NormalTok{ HALO] [{-}{-}scale SCALE\_RATIO\_TO\_FIT\_HALO]}

\ExtensionTok{kpex:}\NormalTok{ KLayout{-}integrated Parasitic Extraction Tool}

\ExtensionTok{Special}\NormalTok{ Options:}
  \ExtensionTok{{-}{-}help,} \AttributeTok{{-}h}\NormalTok{            show this help message and exit}
  \ExtensionTok{{-}{-}version,} \AttributeTok{{-}v}\NormalTok{         show program}\StringTok{\textquotesingle{}s version number and exit}
\StringTok{  {-}{-}log\_level LOG\_LEVEL}
\StringTok{                        log\_level ∈ \{\textquotesingle{}}\NormalTok{all}\StringTok{\textquotesingle{}, \textquotesingle{}}\NormalTok{debug}\StringTok{\textquotesingle{}, \textquotesingle{}}\NormalTok{subprocess}\StringTok{\textquotesingle{}, \textquotesingle{}}\NormalTok{verbose}\StringTok{\textquotesingle{},}
\StringTok{                        \textquotesingle{}}\NormalTok{info}\StringTok{\textquotesingle{}, \textquotesingle{}}\NormalTok{warning}\StringTok{\textquotesingle{}, \textquotesingle{}}\NormalTok{error}\StringTok{\textquotesingle{}, \textquotesingle{}}\NormalTok{critical}\StringTok{\textquotesingle{}\}. Defaults to}
\StringTok{                        \textquotesingle{}}\NormalTok{subprocess}\StringTok{\textquotesingle{}}
\StringTok{  {-}{-}threads NUM\_THREADS}
\StringTok{                        number of threads (e.g. for FasterCap) (default is 48)}

\StringTok{Parasitic Extraction Setup:}
\StringTok{  {-}{-}pdk \{ihp\_sg13g2,sky130A\}}
\StringTok{                        pdk ∈ \{\textquotesingle{}}\NormalTok{ihp\_sg13g2}\StringTok{\textquotesingle{}, \textquotesingle{}}\NormalTok{sky130A}\StringTok{\textquotesingle{}\}}
\StringTok{  {-}{-}out\_dir, {-}o OUTPUT\_DIR\_BASE\_PATH}
\StringTok{                        Output directory path (default is \textquotesingle{}}\NormalTok{output}\StringTok{\textquotesingle{})}

\StringTok{Parasitic Extraction Input:}
\StringTok{  Either LVS is run, or an existing LVSDB is used}

\StringTok{  {-}{-}gds, {-}g GDS\_PATH    GDS path (for LVS)}
\StringTok{  {-}{-}schematic, {-}s SCHEMATIC\_PATH}
\StringTok{                        Schematic SPICE netlist path (for LVS). If none given,}
\StringTok{                        a dummy schematic will be created}
\StringTok{  {-}{-}lvsdb, {-}l LVSDB\_PATH}
\StringTok{                        KLayout LVSDB path (bypass LVS)}
\StringTok{  {-}{-}cell, {-}c CELL\_NAME  Cell (default is the top cell)}
\StringTok{  {-}{-}cache{-}lvs CACHE\_LVS}
\StringTok{                        Used cached LVSDB (for given input GDS) (default is}
\StringTok{                        True)}
\StringTok{  {-}{-}cache{-}dir CACHE\_DIR\_PATH}
\StringTok{                        Path for cached LVSDB (default is .kpex\_cache within}
\StringTok{                        {-}{-}out\_dir)}
\StringTok{  {-}{-}lvs{-}verbose KLAYOUT\_LVS\_VERBOSE}
\StringTok{                        Verbose KLayout LVS output (default is False)}

\StringTok{Parasitic Extraction Options:}
\StringTok{  {-}{-}blackbox BLACKBOX\_DEVICES}
\StringTok{                        Blackbox devices like MIM/MOM caps, as they are}
\StringTok{                        handled by SPICE models (default is False for testing}
\StringTok{                        now)}
\StringTok{  {-}{-}fastercap           Run FasterCap engine (default is False)}
\StringTok{  {-}{-}fastcap             Run FastCap2 engine (default is False)}
\StringTok{  {-}{-}magic               Run MAGIC engine (default is False)}
\StringTok{  {-}{-}2.5D                Run 2.5D analytical engine (default is False)}

\StringTok{Fastercap Options:}
\StringTok{  {-}{-}k\_void, {-}k K\_VOID   Dielectric constant of void (default is 3.9)}
\StringTok{  {-}{-}delaunay\_amax, {-}a DELAUNAY\_AMAX}
\StringTok{                        Delaunay triangulation maximum area (default is 50)}
\StringTok{  {-}{-}delaunay\_b, {-}b DELAUNAY\_B}
\StringTok{                        Delaunay triangulation b (default is 0.5)}
\StringTok{  {-}{-}geo\_check GEOMETRY\_CHECK}
\StringTok{                        Validate geometries before passing to FasterCap}
\StringTok{                        (default is False)}
\StringTok{  {-}{-}diel DIELECTRIC\_FILTER}
\StringTok{                        Comma separated list of dielectric filter patterns.}
\StringTok{                        Allowed patterns are: (none, all, {-}dielname1,}
\StringTok{                        +dielname2) (default is all)}
\StringTok{  {-}{-}tolerance FASTERCAP\_TOLERANCE}
\StringTok{                        FasterCap {-}aX error tolerance (default is 0.05)}
\StringTok{  {-}{-}d\_coeff FASTERCAP\_D\_COEFF}
\StringTok{                        FasterCap {-}d direct potential interaction coefficient}
\StringTok{                        to mesh refinement (default is 0.5)}
\StringTok{  {-}{-}mesh FASTERCAP\_MESH\_REFINEMENT\_VALUE}
\StringTok{                        FasterCap {-}m Mesh relative refinement value (default}
\StringTok{                        is 0.5)}
\StringTok{  {-}{-}ooc FASTERCAP\_OOC\_CONDITION}
\StringTok{                        FasterCap {-}f out{-}of{-}core free memory to link memory}
\StringTok{                        condition (0 = don\textquotesingle{}}\NormalTok{t go OOC, default is 2}\ErrorTok{)}
  \ExtensionTok{{-}{-}auto\_precond}\NormalTok{ FASTERCAP\_AUTO\_PRECONDITIONER}
                        \ExtensionTok{FasterCap} \AttributeTok{{-}ap}\NormalTok{ Automatic preconditioner usage }\ErrorTok{(}\ExtensionTok{default}
                        \ExtensionTok{is}\NormalTok{ True}\KeywordTok{)}
  \ExtensionTok{{-}{-}galerkin}\NormalTok{            FasterCap }\AttributeTok{{-}g}\NormalTok{ Use Galerkin scheme }\ErrorTok{(}\ExtensionTok{default}\NormalTok{ is False}\KeywordTok{)}
  \ExtensionTok{{-}{-}jacobi}\NormalTok{              FasterCap }\AttributeTok{{-}pj}\NormalTok{ Use Jacobi preconditioner }\ErrorTok{(}\ExtensionTok{default}\NormalTok{ is}
                        \ExtensionTok{False}\KeywordTok{)}

\ExtensionTok{Magic}\NormalTok{ Options:}
  \ExtensionTok{{-}{-}magicrc}\NormalTok{ MAGICRC\_PATH}
                        \ExtensionTok{Path}\NormalTok{ to magicrc configuration file }\ErrorTok{(}\ExtensionTok{default}\NormalTok{ is}
                        \StringTok{\textquotesingle{}/foss/pdks/ihp{-}sg13g2/libs.tech/magic/ihp{-}sg13g2.magi}
\StringTok{                        crc\textquotesingle{}}\KeywordTok{)}
  \ExtensionTok{{-}{-}magic\_mode} \DataTypeTok{\{CC}\OperatorTok{,}\DataTypeTok{RC}\OperatorTok{,}\DataTypeTok{R\}}
                        \ExtensionTok{magic\_mode}\NormalTok{ ∈ \{}\StringTok{\textquotesingle{}CC\textquotesingle{}}\NormalTok{, }\StringTok{\textquotesingle{}RC\textquotesingle{}}\NormalTok{, }\StringTok{\textquotesingle{}R\textquotesingle{}}\NormalTok{\}. Defaults to }\StringTok{\textquotesingle{}CC\textquotesingle{}}
  \ExtensionTok{{-}{-}magic\_cthresh}\NormalTok{ MAGIC\_CTHRESH}
                        \ExtensionTok{Threshold} \ErrorTok{(in} \ExtensionTok{fF}\KeywordTok{)} \ControlFlowTok{for}\NormalTok{ ignored }\ExtensionTok{parasitic}\NormalTok{ capacitances}
                        \KeywordTok{(}\ExtensionTok{default}\NormalTok{ is 0.01}\KeywordTok{)}\BuiltInTok{.} \ErrorTok{(}\ExtensionTok{MAGIC}\NormalTok{ command: ext2spice cthresh}
                        \OperatorTok{\textless{}}\NormalTok{value}\OperatorTok{\textgreater{}}\KeywordTok{)}
  \ExtensionTok{{-}{-}magic\_rthresh}\NormalTok{ MAGIC\_RTHRESH}
                        \ExtensionTok{Threshold} \ErrorTok{(in} \ExtensionTok{Ω}\KeywordTok{)} \ControlFlowTok{for}\NormalTok{ ignored }\ExtensionTok{parasitic}\NormalTok{ resistances}
                        \KeywordTok{(}\ExtensionTok{default}\NormalTok{ is 100}\KeywordTok{)}\BuiltInTok{.} \ErrorTok{(}\ExtensionTok{MAGIC}\NormalTok{ command: ext2spice rthresh}
                        \OperatorTok{\textless{}}\NormalTok{value}\OperatorTok{\textgreater{}}\KeywordTok{)}
  \ExtensionTok{{-}{-}magic\_tolerance}\NormalTok{ MAGIC\_TOLERANCE}
                        \ExtensionTok{Set}\NormalTok{ ratio between resistor and device tolerance}
                        \KeywordTok{(}\ExtensionTok{default}\NormalTok{ is 1}\KeywordTok{)}\BuiltInTok{.} \ErrorTok{(}\ExtensionTok{MAGIC}\NormalTok{ command: extresist tolerance}
                        \OperatorTok{\textless{}}\NormalTok{value}\OperatorTok{\textgreater{}}\KeywordTok{)}
  \ExtensionTok{{-}{-}magic\_halo}\NormalTok{ MAGIC\_HALO}
                        \ExtensionTok{Custom}\NormalTok{ sidewall halo distance }\ErrorTok{(in} \ExtensionTok{µm}\KeywordTok{)} \KeywordTok{(}\ExtensionTok{MAGIC}\NormalTok{ command:}
                        \ExtensionTok{extract}\NormalTok{ halo }\OperatorTok{\textless{}}\NormalTok{value}\OperatorTok{\textgreater{}}\KeywordTok{)} \KeywordTok{(}\ExtensionTok{default}\NormalTok{ is no custom halo}\KeywordTok{)}
  \ExtensionTok{{-}{-}magic\_short} \DataTypeTok{\{none}\OperatorTok{,}\DataTypeTok{resistor}\OperatorTok{,}\DataTypeTok{voltage\}}
                        \ExtensionTok{magic\_short}\NormalTok{ ∈ \{}\StringTok{\textquotesingle{}none\textquotesingle{}}\NormalTok{, }\StringTok{\textquotesingle{}resistor\textquotesingle{}}\NormalTok{, }\StringTok{\textquotesingle{}voltage\textquotesingle{}}\NormalTok{\}.}
                        \ExtensionTok{Defaults}\NormalTok{ to }\StringTok{\textquotesingle{}none\textquotesingle{}}
  \ExtensionTok{{-}{-}magic\_merge} \DataTypeTok{\{none}\OperatorTok{,}\DataTypeTok{conservative}\OperatorTok{,}\DataTypeTok{aggressive\}}
                        \ExtensionTok{magic\_merge}\NormalTok{ ∈ \{}\StringTok{\textquotesingle{}none\textquotesingle{}}\NormalTok{, }\StringTok{\textquotesingle{}conservative\textquotesingle{}}\NormalTok{, }\StringTok{\textquotesingle{}aggressive\textquotesingle{}}\NormalTok{\}.}
                        \ExtensionTok{Defaults}\NormalTok{ to }\StringTok{\textquotesingle{}none\textquotesingle{}}

\ExtensionTok{2.5D}\NormalTok{ Options:}
  \ExtensionTok{{-}{-}mode} \DataTypeTok{\{CC}\OperatorTok{,}\DataTypeTok{RC}\OperatorTok{,}\DataTypeTok{R\}}\NormalTok{      mode ∈ \{}\StringTok{\textquotesingle{}CC\textquotesingle{}}\NormalTok{, }\StringTok{\textquotesingle{}RC\textquotesingle{}}\NormalTok{, }\StringTok{\textquotesingle{}R\textquotesingle{}}\NormalTok{\}. Defaults to }\StringTok{\textquotesingle{}CC\textquotesingle{}}
  \ExtensionTok{{-}{-}halo}\NormalTok{ HALO           Custom sidewall halo distance }\ErrorTok{(in} \ExtensionTok{µm}\KeywordTok{)} \ExtensionTok{to}\NormalTok{ override tech}
                        \ExtensionTok{info} \ErrorTok{(}\ExtensionTok{default}\NormalTok{ is no custom halo}\KeywordTok{)}
  \ExtensionTok{{-}{-}scale}\NormalTok{ SCALE\_RATIO\_TO\_FIT\_HALO}
                        \ExtensionTok{Scale}\NormalTok{ fringe ratios, so that halo distance is 100\%}
                        \KeywordTok{(}\ExtensionTok{default}\NormalTok{ is True}\KeywordTok{)}

\ExtensionTok{Environmental}\NormalTok{ variables:}

  \ExtensionTok{Variable}\NormalTok{             Description}
 \ExtensionTok{━━━━━━━━━━━━━━━━━━━━━━━━━━━━━━━━━━━━━━━━━━━━━━━━━━━━━━━━━━━━━━━━━━━━━━━━━━━━}
  \ExtensionTok{KPEX\_FASTCAP\_EXE}\NormalTok{     Path to FastCap2 Executable. Defaults to }\StringTok{\textquotesingle{}fastcap\textquotesingle{}}
  \ExtensionTok{KPEX\_FASTERCAP\_EXE}\NormalTok{   Path to FasterCap Executable. Defaults to }\StringTok{\textquotesingle{}FasterCap\textquotesingle{}}
  \ExtensionTok{KPEX\_KLAYOUT\_EXE}\NormalTok{     Path to KLayout Executable. Defaults to }\StringTok{\textquotesingle{}klayout\textquotesingle{}}
  \ExtensionTok{KPEX\_MAGIC\_EXE}\NormalTok{       Path to MAGIC Executable. Defaults to }\StringTok{\textquotesingle{}magic\textquotesingle{}}
  \ExtensionTok{PDKPATH}\NormalTok{              Optional }\ErrorTok{(}\ExtensionTok{required}\NormalTok{ for default magicrc}\KeywordTok{)}\ExtensionTok{,}\NormalTok{ e.g.}
                       \VariableTok{$HOME}\ExtensionTok{/.volare}
  \ExtensionTok{PDK}\NormalTok{                  Optional }\ErrorTok{(}\ExtensionTok{required}\NormalTok{ for default magicrc}\KeywordTok{)}\ExtensionTok{,} \ErrorTok{(}\ExtensionTok{e.g.}
                       \ExtensionTok{sky130A}\KeywordTok{)}

\end{Highlighting}
\end{Shaded}

Once PEX is succesfully done, the extracted netlist SPICE file will be
generated in the provided output path, which consists of extracted
parasitics which can be further used to do post layout simulations and
verify whether the design layout satisfies the design specifications or
not.

\begin{figure}

\centering{

\pandocbounded{\includegraphics[keepaspectratio]{figures/_fig_pex_outputdir.png}}

}

\caption{\label{fig-output-dir}Extracted Spice file}

\end{figure}%

\subsection{Post layout simulation.}\label{post-layout-simulation.}

We have established testbenches for DC, AC, and transient analysis using
the symbol of the designed inverter as a subcircuit. The next step is to
perform post-layout simulation by replacing the schematic netlist with
the extracted layout netlist (PEX), in order to verify whether the
performance remains consistent after layout parasitics are included.

Firstly we need to open the symbol file in Xschem, edit the properties
of the symbol See Figure~\ref{fig-primitive}. Change from
\texttt{subcircuit} to \texttt{primitive}.

\begin{figure}

\centering{

\pandocbounded{\includegraphics[keepaspectratio]{figures/_fig_primitive.png}}

}

\caption{\label{fig-primitive}Primitive}

\end{figure}%

Once changed to primitive, now we can include the \texttt{.spice} file
from PEX to the testbench. See Figure~\ref{fig-include}

\begin{figure}

\centering{

\pandocbounded{\includegraphics[keepaspectratio]{figures/_fig_include.png}}

}

\caption{\label{fig-include}Include}

\end{figure}%

\subsection{Results}\label{results}

\begin{figure}

\centering{

\pandocbounded{\includegraphics[keepaspectratio]{figures/_fig_pex_output.png}}

}

\caption{\label{fig-result}Result}

\end{figure}%

Figure~\ref{fig-result} shows that the response is as same as we
expected from our schematic.

\section{Xschem Commands}\label{sec-xschem-commands}

Useful shortcuts are as follows

\paragraph{Moving around in a
schematic:}\label{moving-around-in-a-schematic}

\begin{itemize}
\tightlist
\item
  \texttt{Cursor\ keys} to move around
\item
  \texttt{Ctrl-e} to go back to parent schematic
\item
  \texttt{e} to descend into schematic of selected symbol
\item
  \texttt{i} to descend into symbol of selected symbol
\item
  \texttt{f} full zoom on schematic
\item
  \texttt{Shift-z} to zoom in
\item
  \texttt{Ctrl-z} to zoom out
\end{itemize}

\paragraph{Editing schematics:}\label{editing-schematics}

\begin{itemize}
\tightlist
\item
  \texttt{Del} to delete elements
\item
  \texttt{Ins} to insert elements from library
\item
  \texttt{Escape} to abort an operation
\item
  \texttt{Ctrl-\#} to rename components with duplicate names
\item
  \texttt{c} to copy elements
\item
  \texttt{Alt-Shift-l} to add wire label
\item
  \texttt{Alt-l} to add label pin
\item
  \texttt{m} to move selected objects
\item
  \texttt{Shift-R} to rotate selected objects
\item
  \texttt{Shift-F} to mirror / flip selected objects
\item
  \texttt{q} to edit properties
\item
  \texttt{Ctrl-s} to save schematic
\item
  \texttt{t} to place a text
\item
  \texttt{Shift-T} to toggle the \texttt{ignore} flag on an instance
\item
  \texttt{u} to undo an operation
\item
  \texttt{w} to draw a wire
\item
  \texttt{Shift-W} draw wire and snap to close pin or net point
\item
  \texttt{\&} to join, break, and collapse wires
\item
  \texttt{A} to make symbol from schematic
\item
  \texttt{Alt-s} to reload the circuit if changes in a subcircuit were
  made
\end{itemize}

\paragraph{Viewing/Simulating
Schematics}\label{viewingsimulating-schematics}

\begin{itemize}
\tightlist
\item
  \texttt{5} to only view probes
\item
  \texttt{k} to highlight selected net
\item
  \texttt{Shift-K} to unhighlight all nets
\item
  \texttt{Shift-o} to toggle light/dark color scheme
\item
  \texttt{s} to run a simulation
\item
  \texttt{a\ \&\ b} to add cursors to an in-circuit simulation graph
\item
  \texttt{f} full zoom on y- or x-axis in in-circuit simulation graph
\end{itemize}

\section{ngspice Commands}\label{sec-ngspice-commands}

Useful shortcuts are as follows:

\subsection{Commands}\label{commands}

\begin{itemize}
\tightlist
\item
  \texttt{ac\ dec\textbar{}lin\ points\ fstart\ fstop} performs a
  small-signal ac analysis with either linear or decade sweep
\item
  \texttt{dc\ sourcename\ vstart\ vstop\ vincr\ {[}src2\ start2\ stop2\ incr2{]}}
  runs a dc-sweep, optionally across two variables
\item
  \texttt{display} shows the available data vectors in the current plot
\item
  \texttt{echo} can be used to display text, \texttt{\$variable} or
  \texttt{\$\&vector}, can be useful for debugging
\item
  \texttt{let\ name\ =\ expr} to create a new vector;
  \texttt{unlet\ vector} deletes a specified vector; access vector data
  with \texttt{\$\&vec}
\item
  \texttt{linearize\ vec} linearizes a vector on an equidistant time
  scale, do this before an FFT; with \texttt{set\ specwindow=windowtype}
  a proper windowing function can be set
\item
  \texttt{meas} can be used for various evaluations of measurement
  results (see ngspice manual for details)
\item
  \texttt{noise\ v(output\ \textless{}ref\textgreater{})\ src\ (dec\textbar{}lin)\ pts\ fstart\ fstop}
  runs a small-signal noise analysis
\item
  \texttt{op} calculates the operating point, useful for checking bias
  points and device parameters
\item
  \texttt{plot\ expr\ vs\ scale} to plot something
\item
  \texttt{print\ expr} to print it, use \texttt{print\ all} to print
  everything
\item
  \texttt{remzerovec} can be useful to remove vectors with zero length,
  which otherwise cause issues when saving or plotting data
\item
  \texttt{rusage} plot information about resource usage like memory
\item
  \texttt{save\ all} or \texttt{save\ signal} specifies which data is
  saved during simulation; this lowers RAM usage during simulation and
  size of RAW file; do save before the actual simulation statement
\item
  \texttt{setplot} show a list of available plots
\item
  \texttt{set\ var\ =\ value} to set the value of a variable; use
  variable with \texttt{\$var}; \texttt{unset\ var} removes a variable
\item
  \texttt{set\ enable\_noisy\_r} to enable noise of behavioral
  resistors; usually, this is a good idea
\item
  \texttt{shell\ cmd} to run a shell command
\item
  \texttt{show\ :\ param}, like \texttt{show\ :\ gm} shows the \(g_m\)
  of all devices after running an operating point with \texttt{op}
\item
  \texttt{spec} plots a spectrum (i.e.~frequency domain plot)
\item
  \texttt{status} shows the saved parameters and nodes
\item
  \texttt{tf} runs a transfer function analysis, returning transfer
  function, input and output resistance
\item
  \texttt{tran\ tstep\ tstop\ \textless{}tstart\ \textless{}tmax\textgreater{}\textgreater{}}
  runs a transient analysis until \texttt{tstop}, reporting results with
  \texttt{tstep} step size, starting to plot at \texttt{tstart} and
  performs time steps not larger then \texttt{tmax}
\item
  \texttt{wrdata} writes data into a file in a tabular ASCII format;
  easy to further process
\item
  \texttt{write} writes simulation data (the saved nodes) into a RAW
  file; default is binary, can be changed to ASCII with
  \texttt{set\ filetype=ascii}; with \texttt{set\ appendwrite} data is
  added to an existing file
\end{itemize}

\subsection{Options}\label{options}

Use \texttt{option\ option=val\ option=val} to set various options;
important ones are:

\begin{itemize}
\tightlist
\item
  \texttt{abstol} sets the absolute current error tolerance (default is
  1pA)
\item
  \texttt{gmin} is the conductance applied at every node for convergence
  improvement (default is 1e-12); this can be critical for very high
  impedance circuits
\item
  \texttt{klu} sets the KLU matrix solver
\item
  \texttt{list} print the summary listing of the input data
\item
  \texttt{maxord} sets the numerical order of the integration method
  (default is 2 for Gear)
\item
  \texttt{method} set the numerical integration method to \texttt{gear}
  or \texttt{trap} (default is \texttt{trap})
\item
  \texttt{node} prints the node table
\item
  \texttt{opts} prints the option values
\item
  \texttt{temp} sets the simulation temperature
\item
  \texttt{reltol} set the relative error tolerance (default is 0.001 =
  0.1\%)
\item
  \texttt{savecurrents} saves the terminal currents of all devices
\item
  \texttt{sparse} sets the sparse matrix solver, which can run noise
  analysis, but is slower than \texttt{klu}
\item
  \texttt{vntol} sets the absolute voltage error tolerance (default is
  1µV)
\item
  \texttt{warn} enables the printing of the SOA warning messages
\end{itemize}

\subsection{Convergence Helper}\label{convergence-helper}

\begin{itemize}
\tightlist
\item
  \texttt{option\ gmin} can be used to increase the conductance applied
  at every node
\item
  \texttt{option\ method=gear} can lead to improved convergence
\item
  \texttt{.nodeset} can be used to specify initial node voltage guesses
\item
  \texttt{.ic} can be used to set initial conditions
\end{itemize}




\end{document}
